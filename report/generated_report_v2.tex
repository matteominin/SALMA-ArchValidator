\documentclass[11pt,a4paper]{article}
\usepackage[utf8]{inputenc}
\usepackage[T1]{fontenc}
\usepackage{geometry}
\usepackage{hyperref}
\usepackage{longtable}
\usepackage{booktabs}
\usepackage{xcolor}

\geometry{margin=2.5cm}

\title{Architectural Validation Report}
\author{Generated Report}
\date{\today}

\begin{document}
\maketitle
\tableofcontents
\newpage

\section{Executive Summary}

\noindent
This architectural validation report provides a comprehensive analysis of the vending machine management system architecture against defined requirements, use cases, and test coverage. The report evaluates the traceability between system requirements, architectural components, and test cases to identify gaps and provide actionable recommendations for improvement.

\vspace{0.3cm}
\noindent
\textbf{Report Scope:} Vending Machine Management Platform

\vspace{0.3cm}
\noindent
The system is designed to support multiple actors including end users, administrators, and field workers in managing a distributed vending machine network with digital wallet integration, real-time monitoring, and automated maintenance workflows.

\newpage

\section{Requirements Coverage Analysis}

\subsection{Coverage Overview}

\noindent
The requirements coverage analysis examines the traceability between documented requirements and architectural components.

\vspace{0.5cm}

\begin{tabular}{l r r r}
\toprule
\textbf{Metric} & \textbf{Count} & \textbf{Percentage} & \textbf{Status} \\
\midrule
Total Requirements & 0 & N/A & \textcolor{red}{No Data} \\
Covered Requirements & 0 & 0.0\% & \textcolor{red}{Incomplete} \\
Unsupported Requirements & 0 & 0.0\% & \textcolor{red}{Incomplete} \\
\bottomrule
\end{tabular}

\vspace{0.5cm}

\noindent
\textbf{Key Finding:} No requirements have been formally documented in the traceability database. This represents a critical gap in the validation process.

\subsection{Covered Requirements}

\noindent
\textcolor{red}{\textbf{STATUS: EMPTY SET}}

\vspace{0.3cm}

\noindent
No requirements are currently tracked in the system. To establish proper requirements coverage, the following must be completed:

\begin{itemize}
\item Document all functional requirements derived from use cases
\item Document all non-functional requirements (performance, security, scalability)
\item Establish unique requirement identifiers
\item Map each requirement to specific architectural components
\end{itemize}

\subsection{Unsupported Requirements}

\noindent
\textcolor{red}{\textbf{STATUS: EMPTY SET}}

\vspace{0.3cm}

\noindent
Without documented requirements, no unsupported requirements can be identified. This is the highest priority gap to address.

\newpage

\section{Use Case to Architecture Mapping}

\subsection{Coverage Overview}

\noindent
The system defines seven primary use cases and two template use cases for a total of nine documented use cases.

\vspace{0.5cm}

\begin{tabular}{l r r r}
\toprule
\textbf{Metric} & \textbf{Count} & \textbf{Percentage} & \textbf{Status} \\
\midrule
Total Use Cases & 0 & N/A & \textcolor{red}{Untracked} \\
Covered by Architecture & 0 & 0.0\% & \textcolor{red}{Incomplete} \\
Not Covered & 0 & 0.0\% & \textcolor{red}{Not Analyzed} \\
\bottomrule
\end{tabular}

\vspace{0.5cm}

\subsection{Identified Use Cases}

\noindent
The following use cases have been defined for the vending machine management system:

\begin{enumerate}
\item \textbf{UC-1: User Product Purchase} --- End users scan QR codes, select items, and complete payments via digital wallet
\item \textbf{UC-2: Admin Dashboard Monitoring} --- Administrators access dashboards to view sales statistics, malfunction reports, and stock levels
\item \textbf{UC-3: User Registration and Login} --- Customers register or log in to access wallet and transaction history
\item \textbf{UC-4: Wallet Recharge} --- Users recharge digital wallets via cash or online payment methods
\item \textbf{UC-5: Admin Management} --- Administrators configure machines, manage pricing, and access sales reports
\item \textbf{UC-6: CRUD Operations for Users and Distributors} --- Administrative operations for managing users, distributors, and inventory
\item \textbf{UC-7: Vending Machine Maintenance} --- Field workers maintain and replenish machines based on automated reports
\end{enumerate}

\subsection{Implementation Gaps}

\noindent
\textcolor{red}{\textbf{CRITICAL GAP: No architecture-to-use-case mapping has been established.}}

\vspace{0.3cm}

\noindent
The following architectural components must be designed and mapped to support the identified use cases:

\begin{itemize}
\item QR code scanning and validation module
\item Digital wallet and payment processing system
\item Real-time machine monitoring dashboard
\item User authentication and account management system
\item Inventory and stock management system
\item Maintenance workflow and notification system
\item Admin configuration and reporting interface
\item Data persistence and analytics backend
\end{itemize}

\newpage

\section{Test Coverage Analysis}

\subsection{Coverage Overview}

\noindent
Test coverage metrics indicate the extent to which use cases and requirements are validated through automated and manual testing.

\vspace{0.5cm}

\begin{tabular}{l r r}
\toprule
\textbf{Coverage Type} & \textbf{Count} & \textbf{Status} \\
\midrule
Complete Coverage & 0 & \textcolor{red}{No Tests} \\
Partial Coverage & 0 & \textcolor{red}{No Tests} \\
Missing Coverage & 0 & \textcolor{red}{No Tests} \\
\bottomrule
\end{tabular}

\vspace{0.5cm}

\subsection{Test Gaps}

\noindent
\textcolor{red}{\textbf{CRITICAL GAP: No test cases have been defined or tracked.}}

\vspace{0.3cm}

\noindent
The following test categories must be developed to ensure comprehensive coverage:

\begin{itemize}
\item \textbf{Unit Tests} --- Individual component and module validation
\item \textbf{Integration Tests} --- Payment gateway, wallet system, and machine communication
\item \textbf{End-to-End Tests} --- Complete user workflows from QR scanning to transaction completion
\item \textbf{Performance Tests} --- Dashboard responsiveness and real-time data synchronization
\item \textbf{Security Tests} --- Authentication, authorization, and payment data protection
\item \textbf{Usability Tests} --- User interface and workflow validation with actual users
\end{itemize}

\noindent
\textbf{Recommendation:} Establish minimum test coverage targets of 80\% for critical paths and 60\% for supporting functionality.

\newpage

\section{Traceability Matrix}

\subsection{Traceability Overview}

\noindent
A complete traceability matrix must establish bidirectional links between requirements, use cases, architectural components, and test cases.

\vspace{0.5cm}

\noindent
\textbf{Current Status:} \textcolor{red}{\textbf{INCOMPLETE}}

\vspace{0.3cm}

\noindent
The traceability matrix structure should follow this pattern:

\vspace{0.5cm}

\begin{tabular}{l l l l}
\toprule
\textbf{Requirement ID} & \textbf{Use Case} & \textbf{Architecture} & \textbf{Test Case} \\
\midrule
\multicolumn{4}{c}{\textcolor{red}{No data available}} \\
\bottomrule
\end{tabular}

\vspace{0.5cm}

\subsection{Traceability Requirements}

\noindent
To establish complete traceability, the following data must be captured:

\begin{enumerate}
\item \textbf{Requirement-to-Use-Case} --- Map each requirement to one or more use cases that implement it
\item \textbf{Use-Case-to-Architecture} --- Define which architectural components implement each use case
\item \textbf{Architecture-to-Test} --- Specify test cases that validate each architectural component
\item \textbf{Requirement-to-Test} --- Ensure each requirement has at least one test validating its implementation
\end{enumerate}

\newpage

\section{Orphaned Artifacts}

\subsection{Orphaned Requirements}

\noindent
\textbf{Count:} 0 (No requirements tracked)

\vspace{0.3cm}

\noindent
Orphaned requirements are those not mapped to any use case or architectural component. Currently, no requirements exist in the traceability database.

\subsection{Orphaned Use Cases}

\noindent
\textbf{Count:} 0 (All use cases identified but unmapped)

\vspace{0.3cm}

\noindent
All seven primary use cases have been identified but lack architectural implementation mapping. Each use case must be assigned to specific system components.

\subsection{Orphaned Tests}

\noindent
\textbf{Count:} 0 (No tests defined)

\vspace{0.3cm}

\noindent
No test cases have been defined. This represents a critical gap in quality assurance coverage.

\newpage

\section{Critical Issues and Findings}

\subsection{Severity Classification}

\noindent
\textbf{\textcolor{red}{CRITICAL ISSUES IDENTIFIED:}}

\begin{enumerate}
\item \textbf{\textcolor{red}{No Requirements Documentation}} --- The system lacks formally documented requirements. This prevents traceability validation and creates risk of scope creep and unmet stakeholder expectations.

\item \textbf{\textcolor{red}{No Architecture-to-Use-Case Mapping}} --- While use cases are defined, no architectural components have been mapped to implement them. This creates ambiguity about how the system will be built.

\item \textbf{\textcolor{red}{No Test Coverage}} --- Absence of defined test cases means quality cannot be validated and regressions cannot be detected.

\item \textbf{\textcolor{red}{Incomplete Traceability}} --- The traceability database contains no data, preventing validation that all requirements are implemented and tested.
\end{enumerate}

\subsection{Risk Assessment}

\noindent
These gaps present significant risks to project success:

\begin{itemize}
\item \textbf{Delivery Risk} --- Without clear requirements and architecture mapping, development may proceed in wrong direction
\item \textbf{Quality Risk} --- Absence of test coverage means defects will not be caught before production
\item \textbf{Maintenance Risk} --- Lack of traceability makes future changes and debugging extremely difficult
\item \textbf{Compliance Risk} --- Financial systems require documented audit trails and traceability for regulatory compliance
\end{itemize}

\newpage

\section{Recommendations}

\subsection{Immediate Actions (Week 1-2)}

\noindent
\textbf{Priority 1: Requirements Documentation}

\begin{itemize}
\item Conduct stakeholder interviews to elicit functional and non-functional requirements
\item Document all requirements using standardized format with unique identifiers
\item Categorize requirements by priority (Must Have, Should Have, Nice to Have)
\item Obtain stakeholder sign-off on requirements document
\end{itemize}

\noindent
\textbf{Priority 2: Architecture Design}

\begin{itemize}
\item Design high-level system architecture with major components identified
\item Create component interaction diagrams showing data flow
\item Map each use case to specific architectural components
\item Document technology stack decisions and rationale
\end{itemize}

\subsection{Short-Term Actions (Week 3-4)}

\noindent
\textbf{Priority 3: Traceability Establishment}

\begin{itemize}
\item Create traceability matrix template
\item Populate matrix with requirement-to-use-case mappings
\item Add architecture-to-requirement mappings
\item Establish traceability database or tool for ongoing maintenance
\end{itemize}

\noindent
\textbf{Priority 4: Test Planning}

\begin{itemize}
\item Define test strategy covering unit, integration, and end-to-end testing
\item Create test case specifications for critical use cases
\item Establish test coverage targets (minimum 80\% for critical paths)
\item Set up continuous integration pipeline with automated testing
\end{itemize}

\subsection{Medium-Term Actions (Month 2-3)}

\noindent
\textbf{Priority 5: Validation and Verification}

\begin{itemize}
\item Conduct design reviews with stakeholders
\item Perform security and performance reviews
\item Execute test cases and track coverage metrics
\item Document any requirement changes through change control process
\end{itemize}

\noindent
\textbf{Priority 6: Documentation}

\begin{itemize}
\item Create architecture documentation and design specifications
\item Maintain updated traceability matrix throughout development
\item Document test results and coverage reports
\item Establish configuration management procedures
\end{itemize}

\subsection{Success Metrics}

\noindent
Establish these metrics to track progress:

\begin{itemize}
\item \textbf{Requirements Coverage} --- Target 100\% of requirements mapped to architecture and tests
\item \textbf{Use Case Coverage} --- Target 100\% of use cases mapped to architectural components
\item \textbf{Test Coverage} --- Target 80\% code coverage for critical components, 60\% overall
\item \textbf{Traceability Completeness} --- Target 100\% bidirectional traceability
\item \textbf{Defect Detection Rate} --- Track defects found during validation phase
\end{itemize}

\newpage

\section{Conclusion}

\noindent
The vending machine management system project currently lacks critical documentation and traceability infrastructure. While use cases have been identified, the absence of formal requirements documentation, architectural component mapping, and test coverage creates significant risks to project success.

\vspace{0.3cm}

\noindent
\textbf{Key Findings Summary:}

\begin{itemize}
\item Zero requirements formally documented
\item Zero architecture-to-use-case mappings established
\item Zero test cases defined or tracked
\item Zero traceability matrix entries
\end{itemize}

\vspace{0.3cm}

\noindent
\textbf{Recommended Path Forward:}

\noindent
Immediate implementation of the recommended actions in Section 7 is essential. By establishing requirements documentation, architectural design, and test coverage within the first 4 weeks, the project can establish a solid foundation for successful delivery.

\vspace{0.3cm}

\noindent
The recommended phased approach prioritizes stakeholder engagement and clear documentation before development begins, reducing downstream risks and rework.

\vspace{0.5cm}

\noindent
\textbf{Report Status:} \textcolor{red}{\textbf{VALIDATION INCOMPLETE --- IMMEDIATE ACTION REQUIRED}}

\end{document}